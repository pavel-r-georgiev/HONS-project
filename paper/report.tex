\documentclass[bsc,frontabs,twoside,singlespacing,parskip,deptreport,hidel]{infthesis}     % for BSc, BEng etc.
\usepackage{graphicx}
\usepackage[hidelinks]{hyperref}

\begin{document}

\title{Failure Detection over RDMA}

\author{Pavel Georgiev}
\course{Computer Science}


% to choose your report type
% please un-comment just one of the following
%\project{Undergraduate Dissertation} % CS&E, E&SE, AI&L
%\project{Undergraduate Thesis} % AI%Psy
\project{
  \vspace{-1cm}
  \includegraphics[scale=0.45]{logo}\\[1.5cm] 
  4th Year Project Report
  }

\date{\today}

\abstract{

Distributed key-value stores (KVS) are the keystone in a majority of today's data-driven services. 
To optimize their performance and throughput the use of the state-of-the-art networking solution Remote Direct Memory Access (RDMA) is getting more prevelant. 
Furthermore, the number of components in data centers keeps growing, which makes failures more of the expectation rather than the exception.
A failure detection implementation over RDMA is needed, in order to achieve fault tolerance in those circumstances. 

In this paper we explore the implementation of such failure detector.
The paper studies the methods chosen for detecting failure and reaching consensus over a suspected component failure within distributed setting.
It also outlines the changes made to integrate the use of RDMA for communication within the implementation and analyses the performance of the finished implementation.
}

\maketitle

\section*{Acknowledgements}
Acknowledgements go here. 

\tableofcontents

% \pagenumbering{arabic}


\chapter{Introduction}

Many of the leading companies in the digital space, including Google, Amazon and Facebook, use distributed key-value stores (KVS) to successfully scale their solutions to a global level.



\section{Goals}

Divide your chapters into sub-parts as appropriate.

\section{Contributions}
Contributions done:
\begin{itemize}
\item
1
\item
2
\item
3.
\item
4
\item
5
\end{itemize}

\section{Report outline}


\chapter{Background}
\section{Failure Detection}
\section{RDMA}

\chapter{Failure detector implementation}
\section{Testing framework}
\section{Heartbeats}
\section{Adaptive timeout}
\section{PAXOS}

\chapter{Failure detection over RDMA}

\chapter{Integration with existing KVS}
\chapter{Analysis}

Of course
you may want to use several chapters and much more text than here.

% use the following and \cite{} as above if you use BibTeX
% otherwise generate bibtem entries
\bibliographystyle{plain}
\bibliography{bibliography}

\end{document}
